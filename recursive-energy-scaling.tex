\documentclass{article}
\usepackage{amsmath, amssymb, amsthm}

\title{Recursive Energy Scaling Theorem}
\author{Unified Theory of Energy Framework}
\date{\today}

\begin{document}

\maketitle

\begin{abstract}
This theorem formalizes the recursive transitions of energy across Degrees of Surface Interaction ($D$). It establishes the conditions under which energy expands to a higher dimension ($D+1$) or compacts into a lower dimension ($D-1$), maintaining conservation while defining dimensional stability. The theorem follows directly from Theorem 1 and Definitions 1-4 of the Unified Theory of Energy.
\end{abstract}

\section{Preliminaries}

\subsection{Energy States and Their Interactions}
We recall that energy exists in three states:

\begin{itemize}
    \item \textbf{Radiation} ($R$): Energy extended outwardly, intended for absorption.
    \item \textbf{Gravitation} ($G$): Energy stored within mass; absorbed radiation and potential energy.
    \item \textbf{Particulate Motion} ($I$): Inertial energy affecting mass and gravitation while being affected by radiation.
\end{itemize}

These states cannot exist in isolation; they must interact and transform recursively.

\subsection{Mass Structure and Energy Recursion}
\textbf{Definition 4}: A Mass Structure is a collection of Particles drawn together by Gravitation while held apart by Radiation and Particulate Motion.

This implies:
\begin{enumerate}
    \item Gravitation ($G$) binds energy into mass.
    \item Particulate Motion ($I$) balances gravitation within the mass structure.
    \item Radiation ($R$) extends energy outward, resisting total collapse.
\end{enumerate}

Since a mass structure is always engaged in energy storage and exchange, it follows that energy transitions must 	extbf{scale** recursively across increasing or decreasing Degrees of Surface Interaction.

\section{Recursive Energy Scaling Theorem}

\begin{theorem}[Recursive Energy Scaling Theorem]
Energy transitions recursively across Degrees of Surface Interaction $D \in \mathbb{N}$, with transformations between dimensions governed by the relative dominance of Radiation $(R)$, Gravitation $(G)$, and Particulate Motion $(I)$. These transitions define dimensional stability and transformation as follows:

\begin{enumerate}
    \item \textbf{Expansion to Higher Dimension ($D+1$)}:
    \begin{equation}
        \text{If } R + I > G, \quad \text{then } D \to D+1.
    \end{equation}
    Energy extends, forming larger recursive structures (e.g., atomic bonds $\rightarrow$ molecular networks $\rightarrow$ biological complexity).
    
    \item \textbf{Compression to Lower Dimension ($D-1$)}:
    \begin{equation}
        \text{If } G > R + I, \quad \text{then } D \to D-1.
    \end{equation}
    Energy compacts, leading to condensation, solidification, or collapse into stored potential energy.
    
    \item \textbf{Equilibrium Condition (Stable Dimension)}:
    \begin{equation}
        R + I = G \quad \Rightarrow \quad D \text{ remains stable.}
    \end{equation}
    This represents a 	extbf{stable energy configuration** where mass structures maintain recursive equilibrium.
\end{enumerate}

\end{theorem}

\section{Implications and Dimensional Mapping}

The above theorem provides a direct mathematical criterion for 	extbf{when and how energy transforms across dimensions**. Applying this theorem to known degrees of surface interaction:

\begin{itemize}
    \item $D=0$: Absorption and shedding of radiation (pure radiation state).
    \item $D=1$: Particulate Motion (charge, current).
    \item $D=2$: Bonding interactions, atomic and molecular structuring.
    \item $D=3$: Structural formation, gas expansion, mass aggregation.
    \item $D=4$: Unicellular life, self-replicating structures.
    \item $D=5$: Multicellular life, specialized cellular interactions.
    \item $D=6$: Higher-order cognition, recursive awareness.
\end{itemize}

\section{Conclusion}

This theorem establishes a recursive model where energy transformations determine dimensional transitions. Unlike static models, this framework introduces a 	extbf{dynamic recursion of energy scaling**, unifying radiation, gravitation, and particulate motion across increasing levels of complexity.

\end{document}
